\section{Machine Description}

We ran all of the benchmark's on a server, so its hardware
specification is the most relevant.  Before we detail the configuration of our hardware, there are three things worth mentioning here.
First, we used QEMU with the CPU type
set to \verb|host,+rdtscp,+invtsc| in order to ensure that the host's
TSC resources were passed through to the guest correctly. 
Next, we also
used the Linux \verb|isolcpus| boot option and task CPU affinity infrastructure to
dedicate a core to the guest OS.
Finally, we used the machine's BIOS to disable hyperthreading and
dynamic frequency scaling to keep the CPU performance characteristics repeatable.
Below we report on the characteristics of our hardware:
\begin{enumerate}
    \item Processor: an Intel i7-4770 CPU with 4 cores, designed for a base clock of 3.40 GHz. 
    It has 32K of L1 cache, 256K of L2 cache and 8192K of L3 cache.
    \item Memory bus: 64 bits
    \item I/O bus: 64 bits
    \item RAM: 12GB DDR3 RAM with a memory clock speed of 1600 MT/s and data width of 64 bits.
    \item Disk: 1TB SATA hard drive with a rotation rate of 7200 RPM and physical sector size of 4MB.
    \item Network card speed: 1000Mb/s
    \item Operating system: 18.04.4 LTS (Bionic Beaver)
\end{enumerate}

\section{Introduction}
The goal of this project is to measure the performance of \textit{seL4},
a formally verified kernel ~\cite{klein2009sel4}. 
We are a group of two students: Peter Amidon and Enze ``Alex'' Liu. 

We worked together on the experiments. Peter set up the boilerplate to
get a root task working and memory mapped together, and both of us wrote the
benchmark code.

We implemented all of the measurements in C, using the rdtscp
instruction to measure time.  We used GCC 7.4.0 with both -O2 and -g
to compile the code.

Since seL4 is difficult to run on real hardware, due to its extremely
limited device driver support, we decided to run seL4 inside of QEMU,
at least for this phase of the project. In order to make timing more
accurate, we ran QEMU with KVM hardware virtualization, since as far
as we could tell, using QEMU with the tiny code generator emulated
processor would result in a very strange cost model (instruction costs
are very odd compared to hardware, and the TSC is essentially passed
through). In order to minimize interference from the host system, we
used the Linux \verb|isolcpus| and task CPU affinity infrastructure to
dedicate a core to running the virtual machine; we also disabled
hyperthreading to reduce interference from other threads. We hope that
this will significantly reduce the extra variance introduced by
virtualization as opposed to running directly on native hardware.

\section{Operations}

\subsection{CPU, Scheduling, and OS Services}
We report on our results below.
\subsubsection{Measurement Overhead}
We performed ten rounds of experiments, with each round sampling the measurement overhead 100,000 times. We only report the average overhead across 100,000 samples for each round.
Table \ref{tab:t1} shows the average overhead across ten rounds. 
The results are very consistent.
All ten experiments got an average measurement overhead of 31 cycles.

\begin{table}[htb]

\caption{Mean, min and max values of average measurement overhead for ten rounds}

    \begin{tabular}{|c|c|c|} 
     \hline
     Mean & Min & Max \\ 
     \hline
     31 & 31 & 31 \\ 
     \hline
    \end{tabular}
    \label{tab:t1}
\end{table}

\subsubsection{Call Overhead}
For each function with a different number of args, we performed ten rounds of experiments, with each round sampling the overhead 100,000 times. We only report the average overhead across 100,000 samples for each round.
Table \ref{tab:t2} shows the average overhead for ten rounds for each function. 
We see the average overhead increases with the number of args. 
We also observ that most of the times the average overhead is stable.

\begin{table}[htb]

\caption{Mean, min and max values of average call overhead for ten rounds}

    \begin{tabular}{|c|c|c|c|} 
     \hline
     \# of Args & Mean & Min & Max \\ 
     \hline
     0 & 34 & 34 & 34 \\ 
     \hline
     1 & 34 & 34 & 34 \\ 
     \hline
     2 & 36 & 36 & 36 \\ 
     \hline
     3 & 36.7 & 36 & 39 \\ 
     \hline
     4 & 39 & 39 & 39 \\ 
     \hline
     5 & 40 & 40 & 40 \\ 
     \hline
     6 & 41.2 & 40 & 43 \\ 
     \hline
     7 & 42.9 & 42 & 43 \\ 
     \hline
    \end{tabular}
    \label{tab:t2}
\end{table}


\subsubsection{Syscall Overhead}
We performed ten rounds of experiments, with each round sampling the overhead 100,000 times. We only report the average overhead across 100,000 samples for each round.
Table \ref{tab:t3} shows the average syscall overhead for ten rounds. 

\begin{table}[tb]

    \caption{Mean, min and max values of syscall overhead for ten rounds}
    
    \begin{tabular}{|c|c|c|} 
        \hline
        Mean & Min & Max \\ 
        \hline
        1681.3 & 1678 & 1699 \\ 
        \hline
       \end{tabular}
        \label{tab:t3}
\end{table}

    
\subsubsection{Task Creation Time}
\label{subsubsec:tct}
We performed ten rounds of experiments, with each round sampling the overhead 100,000 times. We only report the average overhead across 100,000 samples for each round.
Table \ref{tab:t4} shows the average task creation time for ten rounds. We observed some obvious outliers in this case. However, we are not able to explain what happened.

\begin{table}[htb]

    \caption{Mean, min and max values of task creation time for ten rounds}
\begin{tabular}{|c|c|c|} 
    \hline
    Mean & Min & Max \\ 
    \hline
    67620.9 & 24655 & 454173 \\ 
    \hline
   \end{tabular}
    \label{tab:t4}
\end{table}

\subsubsection{Task Switch Time}
We performed ten rounds of experiments, with each round sampling the overhead 100,000 times. We only report the average overhead across 100,000 samples for each round.
Table \ref{tab:t5} shows the average task switch time for ten rounds. Similar to section \ref{subsubsec:tct}, we observed some outliers, yet unable to explain why.



\begin{table}[htb]

    \caption{Mean, min and max values of task switch time for ten rounds}
 
\begin{tabular}{|c|c|c|} 
    \hline
    Mean & Min & Max \\ 
    \hline
    6458.8 & 2158 & 45110 \\ 
    \hline
   \end{tabular}
    \label{tab:t5}
\end{table}
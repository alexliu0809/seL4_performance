\subsection{CPU, Scheduling, and OS Services}
We report on our results below.
\subsubsection{Measurement Overhead}
We performed ten rounds of experiments, with each round sampling the measurement overhead 100,000 times. We only report the average overhead across 100,000 samples for each round.
Table \ref{tab:t1} shows the average overhead across ten rounds. 
The results are very consistent.
All ten experiments got an average measurement overhead of 31 cycles.

\begin{table}[htb]

\caption{Mean, min and max values of average measurement overhead for ten rounds}

    \begin{tabular}{|c|c|c|c|} 
     \hline
     Mean & Min & Max & Stddev\\ 
     \hline
     31 & 31 & 31 & 0\\ 
     \hline
    \end{tabular}
    \label{tab:t1}
\end{table}

\subsubsection{Call Overhead}
For each number of arguments, we performed ten rounds of experiments,
with each round sampling the overhead 100,000 times.
%
We only report the average overhead across 100,000 samples for each
round.
%
Table \ref{tab:t2} shows the average overhead for ten rounds for each
function.
%
We see the average overhead increases with the number of args. We also
observe that most of the time, the average overhead is stable.

\begin{table}[htb]

\caption{Mean, min and max values of average call overhead for ten rounds}

    \begin{tabular}{|c|c|c|c|c|} 
     \hline
     \# of Args & Mean & Min & Max & Stddev \\ 
     \hline
     0 & 34 & 34 & 34 & 0 \\ 
     \hline
     1 & 34 & 34 & 34 & 0\\ 
     \hline
     2 & 36 & 36 & 36 & 0\\ 
     \hline
     3 & 36.7 & 36 & 39 & 1.02\\ 
     \hline
     4 & 39 & 39 & 39 & 0\\ 
     \hline
     5 & 40 & 40 & 40 & 0\\ 
     \hline
     6 & 41.2 & 40 & 43 & 0.75 \\ 
     \hline
     7 & 42.9 & 42 & 43 & 0.30\\ 
     \hline
    \end{tabular}
    \label{tab:t2}
\end{table}


\subsubsection{Syscall Overhead}
We performed ten rounds of experiments, with each round sampling the
overhead 100,000 times.
%
We only report the average overhead across 100,000 samples for each
round.
%
Table \ref{tab:t3} shows the average and standard deviation for the
syscall overhead over the ten rounds.

\begin{table}[tb]

    \caption{Mean, min and max values of syscall overhead for ten rounds}
    
    \begin{tabular}{|c|c|c|c|} 
        \hline
        Mean & Min & Max & Stddev\\ 
        \hline
        1681.3 & 1678 & 1699 & 5.98\\ 
        \hline
       \end{tabular}
        \label{tab:t3}
\end{table}

    
\subsubsection{Task Creation Time}
\label{subsubsec:tct}
We performed ten rounds of experiments, with each round sampling the
overhead 100,000 times. %
We only report the average overhead across 100,000 samples for each
round. %
Table \ref{tab:t4} shows the average and standard deviation of task
creation time for the ten rounds. %
We observed some obvious outliers in this case, where the timestamp
counter appeared to move backwards. %
However, we are not able to explain what happened.

\begin{table}[htb]

    \caption{Mean, min and max values of task creation time for ten rounds}
\begin{tabular}{|c|c|c|c|} 
    \hline
    Mean & Min & Max & Stddev\\ 
    \hline
    67620.9 & 24655 & 454173 & 128850.70\\ 
    \hline
   \end{tabular}
    \label{tab:t4}
\end{table}

\subsubsection{Task Switch Time}
We performed ten rounds of experiments, with each round sampling the
overhead 100,000 times. %
We only report the average overhead across 100,000 samples for each
round. %
Table \ref{tab:t5} shows the average and standard deviation of the
task switch time over the ten rounds. %
As with section~\ref{subsubsec:tct}, we observed some outliers, but
were unable to explain their cause.

\begin{table}[htb]

    \caption{Mean, min and max values of task switch time for ten rounds}
 
\begin{tabular}{|c|c|c|c|} 
    \hline
    Mean & Min & Max & Stddev\\ 
    \hline
    6458.8 & 2158 & 45110 & 12883.74\\ 
    \hline
   \end{tabular}
    \label{tab:t5}
\end{table}

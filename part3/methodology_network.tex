\subsection{Round Trip Time}
We build our network stack with two libraries, \textit{ethdrivers}~\cite{utillibs55:online} and \textit{lwip}~\cite{lwIPWiki65:online}.
Despite many limitations and little documentation, 
we managed to get UDP packet sending working. 
We were not able to register interrupt request (aka IRQ) properly. 
Thus, the interrupt request from network card would never be delivered.
However, 
we came up with a workaround that allowed us to know when an UDP packet was received.
We achieved this by consistently polling from the network driver. 
A callback function will be called when a packet arrived.\\
\\
We now describe the experiment we performed.
We measured the round trip time of a UDP packet from a client (our program) to an echo server on localhost.
We start the timer when we send the packet.
As mentioned above, to get notified when the echo packet comes back,
we have to consistently poll and wait for a callback function to be called.
We stop polling when the callback function is called and and the timer.\\
\\
For one run, we sent UDP packets to our echo server 10 times, 
and recorded the average round trip time.
We then reported on the average reading/writing time over 10 runs.
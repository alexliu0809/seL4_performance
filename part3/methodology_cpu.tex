\subsection{Measuring Overhead in CPU, Scheduling, and OS Services}
%
In this section, we are mostly concerned with the overhead incurred in
CPU, scheduling, and OS services.
%
We selected \textit{rdtscp} function as our timestamp function as
recommended by the instructor.
%
We then placed whichever function we wanted to evaluate in between of two \textit{rdtscp} functions.
%
Apparently, if nothing were placed in between, we would get an estimate of the \textit{measurement overhead}, 
the overhead introduced by the timestamp function.
%
When measuring the overhead of other functions, we would need to
subtract the \textit{measurement overhead} to get the real overhead
introduced by that function.
%
For each of the tests, we ran 10 sample runs, and then computed the
mean and the standard deviation over all the sample runs.
%

For each of the tests except process creation, we ran 100000 of the
operations inside a single sample run.
%
Since process creation was an order of magnitude slower than anything
else, we ran only 10000 process creation operations per run.
%
In every case, except syscall overhead, we achieved this within each
run by running the requisite number of individual timing operations
and summing them: for measuring overhead, the only way to do it
involves running the timing measurement; and for benchmarking
procedure calls and thread switches, it is necessary to record the
time after the call/thread switch has happened without waiting for a
return/context switch back to happen.
%
Therefore, the measurement overhead is present in all of these
measurements, but for the syscall measurement, it is minimal (as it
occurs only once, when the syscall is run 100000 times).

We implemented most of the measurements in a straightforward fashion.
%
For the syscall overhead measurement, we used the syscall
\verb|seL4_BenchmarkNullSyscall| which is an empty syscall designed
for benchmarking system call transition overhead.
%
For the two measurements involving multiple tasks, we measured the
time to set up a new thread control block and/or allow a context
switch to the new thread, since this is the only primitive notion of
task in seL4.
%
In the future, we may also consider implementing a notion of processes
on top of these primitive threads, allowing benchmarking those as
well; however, since processes are essentially a composite of ``new
virtual address space'', ``new capability space'', and ``new thread'',
we believe it is likely more interesting to measure each of these
individually.

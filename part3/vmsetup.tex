\section{Virtual Machine Setup}
We setup a kvm on the machine described in section \ref{sec:md}.
In this section, 
we detail how we setup and run the virtual machine and 
what commands you need to run our code.

\subsection{Install Dependencies}
To set up host machine to build and run seL4, 
we need to install the required dependencies.
Examples of dependencies include \textit{cmake, ccache,} and \textit{ninja-build}.
A comprehensive list of dependencies required can be found here ~\cite{HostDepe94:online}. 
For repeatability, we also note the installed version of some critical dependencies here.
\begin{enumerate}
    \item cmake version 3.16.5
    \item ninja version 1.8.2
    \item python3 version 3.6.9
    \item QEMU version 2.11.1
    \item gcc version 7.4.0    
\end{enumerate}

\subsection{Link Extra Libraries}
After installing all the necessary dependencies, 
one could create and build a project following the tutorial here ~\cite{Tutorial63:online}. 
However, 
the default project does not incorporate some of the imperative
libraries. So we have to link them first before building the project. More specifically, we need to change one line in the project's \textit{CMakeList.txt} file.
We have to add three libraries, namely \textit{ethdrivers}, \textit{lwip} and \textit{pci}, 
to \textit{target\_link\_libraries}. 
This ensures that we could use these libraries in later phases.

\subsection{Build the Project}
After linking the important libraries, 
we can now build the project.
To build the project on our specific platform, 
we have to supply the following arguments when building.
\begin{lstlisting}
    -DPLATFORM=x86_64
    -DSIMULATION=1
    -DKernelX86MicroArch=nehalem
    -DKernelBenchmarks=generic
    -DLibLwip=ON    
\end{lstlisting}

\subsection{Start Simulation}
Now we've built the project successfully. 
We can start running our code in QEMU, a simulated environment. 
To start simulation properly, we need to pass the following arguments.

\begin{lstlisting}
--cpu=host,+rdtscp,+invtsc
--extra-qemu-args='-enable-kvm\
-netdev user,id=net0,net=192.168.5.0/24\
-device virtio-net-pci,netdev=net0\
-object filter-dump,id=f1,netdev=net0,\
file=dump.dat -d int'
\end{lstlisting}